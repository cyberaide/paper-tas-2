\section{Introduction}

Science advancement and engineering discoveries have been requiring increasingly large amount of computing power, and many of them usually need very large-scale resources that cannot be efficiently managed by individual research groups. Dedicated big computing facilities play an important role here, in which resources are shared among various groups of researchers, while the facilities themselves are managed by dedicated staff. XSEDE, an evolution from the TeraGrid, for instance, provides large scale resources to researchers to advance science and expedite engineering discoveries \cite{www-xsede}. It is open to eligible researchers, while a research proposal submission is required. Upon approval of the proposal the researcher is granted a predefined amount of resources, e.g., computing core-hours, storage spaces, and support by staff. The resources represent a tremendous amount of investment, thus some natural questions try to figure out how such resources help: 

\begin{enumerate}
\item Is there a way to measure the impact of providing such facilities like XSEDE to scientist? 

\item How is the impact of scientific research of individual user, project, or field of study, with respect to the resources allocated? 

\item When evaluating a proposal request, what is the criteria to judge whether the proposal is potentially leading to good research and broader impact, and how to get metrics to back up this?
\end{enumerate}

To answer these questions, we need a process to quantify the scientific outcome for the individual research units, and then define metrics when correlating to the consumed resources to finally measure the impacts of the science activities conducted. In this paper, we present a framework that does this, while studying activities and their outcomes on XSEDE. For this paper we restrict our effort to those related to scientific publications as the base unit of the research output, and obtained data as well as derived various metrics on top of that to measure the impact of individual users, projects, Field of Science (FOS), and XSEDE itself in a whole.

In the following sections we will first briefly discuss the related works, and then present our designed framework and some implementation details. The results and discussions then follow. Finally, we outline our future plans and conclude the paper.

