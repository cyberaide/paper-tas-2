\section{Discussion and Future Work}

This paper does not address the name ambiguity issue, which deserves dedicated research and in fact numerous research has been done tackling this issue. The root cause of this issue is that the metadata of the publications simply does not include enough information to distinguish similar user names. This is not a problem specific to our study but for the automated bibliometrics analysis in general, e.g., Google Scholar also include false positive publications in user profile but leave it to the user to curate the results to make it more accurate. In the future we will try to tackle the problem based on other available data - field of science, organization, funding data, co-author relationship etc. while conducting unsupervised machine learning techniques like k-means clustering.

As the ultimate approach is to let users vetting their publication list, we would try to include more such data. One pathway is to work with the XSEDE portal team while providing the publication data we have collected as a suggestion service, in the hope to provide more convenient way for users to quickly populate the vetted publications library. This is a work in progress now.

We have also started another similar activity, in which we wanted to extracting and parsing the publication data from past TeraGrid/XSEDE quarterly reports. These data, while not curated on per user basis, do have project level association information thus could serve quite well for most of our analysis.

Another activity we are also conducting is social networking related analysis among publications, users, projects, FOSes, etc. based on citation and co-authorship relations.
